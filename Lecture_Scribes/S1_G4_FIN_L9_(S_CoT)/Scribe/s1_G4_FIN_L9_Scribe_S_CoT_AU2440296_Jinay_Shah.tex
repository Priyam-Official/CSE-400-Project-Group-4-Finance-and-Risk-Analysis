\documentclass[12pt]{article}
\usepackage{amsmath,amssymb}
\usepackage{geometry}
\geometry{margin=1in}

\begin{document}

\begin{center}
\textbf{CSE 400: Fundamentals of Probability in Computing}\\
\textbf{Lecture 9: Continuous Random Variables}\\
(Uniform and Exponential Random Variables)
\end{center}

\section{Types of Continuous Random Variables}

This lecture covers the following types of continuous random variables:

Uniform random variable

Exponential random variable

\section{Continuous Random Variable}

A continuous random variable is characterized by:

A probability density function (PDF), denoted $f_X(x)$

A cumulative distribution function (CDF), denoted $F_X(x)$

The probability of an event is computed using the PDF over an interval.

\section{Uniform Random Variable}

\subsection{Definition (PDF)}

Let $X$ be a uniform random variable on the interval $[a,b)$.

The probability density function is:
\[
f_X(x)=
\begin{cases}
\dfrac{1}{b-a}, & a \le x < b \\
0, & \text{elsewhere}
\end{cases}
\]

\subsection{Definition (CDF)}

The cumulative distribution function is:
\[
F_X(x)=
\begin{cases}
0, & x<a \\
\dfrac{x-a}{b-a}, & a \le x < b \\
1, & x \ge b
\end{cases}
\]

\subsection{Graphical Interpretation}

PDF graph: a constant horizontal line at height $\dfrac{1}{b-a}$ between $a$ and $b$; zero elsewhere.

CDF graph:

Flat at 0 for $x<a$

Linearly increasing from 0 to 1 over $[a,b)$

Flat at 1 for $x \ge b$

\section{Example \#1: Uniform Random Variable}

\textbf{Problem Statement}

The phase of a sinusoid, $\Theta$, is uniformly distributed over $[0,2\pi)$.

The PDF is:
\[
f_\Theta(\theta)=
\begin{cases}
\dfrac{1}{2\pi}, & 0 \le \theta < 2\pi \\
0, & \text{otherwise}
\end{cases}
\]

\textbf{General Property Used}

For a uniform random variable on $[0,2\pi)$:
\[
\Pr(a<\Theta<b)=\dfrac{b-a}{2\pi}
\]

\subsection*{(a) Find $\Pr(\Theta>\frac{3\pi}{4})$}

\[
\Pr\left(\Theta>\frac{3\pi}{4}\right)
=
\dfrac{2\pi-\frac{3\pi}{4}}{2\pi}
=
\dfrac{\frac{5\pi}{4}}{2\pi}
=
\dfrac{5}{8}
\]

\subsection*{(b) Find $\Pr(\Theta<\pi \mid \Theta>\frac{3\pi}{4})$}

Let
\[
A=\{\Theta<\pi\}, \quad B=\left\{\Theta>\frac{3\pi}{4}\right\}.
\]

Using conditional probability:
\[
\Pr(A\mid B)=\dfrac{\Pr(A\cap B)}{\Pr(B)}
\]

Compute numerator:
\[
\Pr\left(\frac{3\pi}{4}<\Theta<\pi\right)
=
\dfrac{\pi-\frac{3\pi}{4}}{2\pi}
=
\dfrac{\frac{\pi}{4}}{2\pi}
=
\dfrac{1}{8}
\]

From part (a):
\[
\Pr(B)=\dfrac{5}{8}
\]

Thus:
\[
\Pr\left(\Theta<\pi \mid \Theta>\frac{3\pi}{4}\right)
=
\dfrac{1/8}{5/8}
=
\dfrac{1}{5}
\]

\subsection*{(c) Find $\Pr(\cos\Theta<\frac{1}{2})$}

Solve:
\[
\cos\Theta=\frac{1}{2} \Rightarrow \Theta=\frac{\pi}{3}, \frac{5\pi}{3}
\]

Thus:
\[
\cos\Theta<\frac{1}{2} \quad \text{for} \quad \frac{\pi}{3}<\Theta<\frac{5\pi}{3}
\]

Compute probability:
\[
\Pr\left(\cos\Theta<\frac{1}{2}\right)
=
\dfrac{\frac{5\pi}{3}-\frac{\pi}{3}}{2\pi}
=
\dfrac{\frac{4\pi}{3}}{2\pi}
=
\dfrac{2}{3}
\]

\section{Uniform Random Variable: Application Examples}

Phase of a sinusoidal signal when all phase angles between $0$ and $2\pi$ are equally likely.

Random number generated by a computer between 0 and 1 for simulations.

Arrival time of a user within a known time window, assuming no time preference.

\section{Exponential Random Variable}

\subsection{Definition (PDF)}

For parameter $b>0$, the exponential random variable has PDF:
\[
f_X(x)=\dfrac{1}{b}\exp\left(-\dfrac{x}{b}\right)u(x)
\]
where $u(x)$ is the unit step function.

\subsection{Definition (CDF)}

\[
F_X(x)=\left[1-\exp\left(-\dfrac{x}{b}\right)\right]u(x)
\]

\subsection{Graphical Interpretation}

PDF graph: decreasing exponential curve starting at $\dfrac{1}{b}$ at $x=0$, approaching 0 as $x \to \infty$.

CDF graph: increasing curve starting at 0 and asymptotically approaching 1.

(Example plot shown for $b=2$)

\section{Example \#2: Exponential Random Variable}

\textbf{Problem Statement}

Let $X$ be an exponential random variable with PDF:
\[
f_X(x)=e^{-x}u(x)
\]

\subsection*{(a) Find $\Pr(3X<5)$}

Rewrite the event:
\[
3X<5 \Rightarrow X<\frac{5}{3}
\]

Compute probability using the CDF:
\[
\Pr\left(X<\frac{5}{3}\right)=\int_0^{5/3} e^{-x}\,dx
\]

Evaluate:
\[
=[-e^{-x}]_0^{5/3}
=
1-e^{-5/3}
\]

\subsection*{(b) Generalize to find $\Pr(3X<y)$ for arbitrary constant $y$}

Rewrite:
\[
3X<y \Rightarrow X<\frac{y}{3}
\]

Thus:
\[
\Pr(3X<y)=\Pr\left(X<\frac{y}{3}\right)
\]

Using the CDF:
\[
=1-e^{-y/3}
\]

\end{document}