\documentclass[12pt]{article}
\usepackage{amsmath,amssymb}
\usepackage{geometry}
\usepackage{setspace}
\usepackage{listings}

\geometry{margin=1in}
\setstretch{1.2}

\begin{document}

\begin{center}
\Large \textbf{CSE 400: Fundamentals of Probability and Random Variables} \\
\large \textbf{Lecture: Types of Continuous Random Variables} \\
SEAS, Ahmedabad University
\end{center}

\hrule
\vspace{1em}

\section*{Types of Continuous Random Variables}

\section*{1. Uniform Random Variable}

\subsection*{Definition}

A continuous random variable $(\Theta)$ is said to be uniformly distributed over an interval $[a,b]$ if its probability density function is constant over that interval.

\subsection*{Support}

\[
a \le \Theta < b
\]

\subsection*{Probability Density Function (PDF)}

\[
f_\Theta(\theta)=
\begin{cases}
\dfrac{1}{b-a}, & a \le \theta < b \\
0, & \text{otherwise}
\end{cases}
\]

\subsection*{Given}

\[
\Theta \sim \text{Uniform}[0,2\pi]
\]

\[
f_\Theta(\theta)=
\begin{cases}
\dfrac{1}{2\pi}, & 0 \le \theta < 2\pi \\
0, & \text{otherwise}
\end{cases}
\]

\subsection*{(a)}

\[
\Pr(\Theta > \tfrac{3\pi}{4})
\]

\[
= \int_{3\pi/4}^{2\pi} \frac{1}{2\pi} \, d\theta
\]

\[
= \frac{1}{2\pi}\left(2\pi - \frac{3\pi}{4}\right)
\]

\[
= \frac{5}{8}
\]

\subsection*{(b)}

\[
\Pr(\Theta < \pi \mid \Theta > \tfrac{3\pi}{4})
\]

\[
\Pr(A\mid B)=\frac{\Pr(A\cap B)}{\Pr(B)}
\]

\[
A=\{\Theta<\pi\}, \quad B=\{\Theta>\tfrac{3\pi}{4}\}
\]

\[
A\cap B=\left\{\tfrac{3\pi}{4}<\Theta<\pi\right\}
\]

\[
\Pr(A\cap B)
=\int_{3\pi/4}^{\pi}\frac{1}{2\pi}\,d\theta
=\frac{1}{8}
\]

\[
\Pr(B)=\frac{5}{8}
\]

\[
\Pr(\Theta<\pi\mid\Theta>\tfrac{3\pi}{4})
=\frac{1/8}{5/8}
=\frac{1}{5}
\]

\subsection*{(c)}

\[
\Pr(\cos(\Theta)<\tfrac{1}{2})
\]

\[
\cos\theta=\frac{1}{2}
\Rightarrow \theta=\frac{\pi}{3},\frac{5\pi}{3}
\]

\[
\Pr\left(\frac{\pi}{3}<\Theta<\frac{5\pi}{3}\right)
=\int_{\pi/3}^{5\pi/3}\frac{1}{2\pi}\,d\theta
\]

\[
=\frac{1}{2\pi}\left(\frac{4\pi}{3}\right)
=\frac{2}{3}
\]

\section*{2. Exponential Random Variable}

\subsection*{Definition}

A random variable $(X)$ is exponential if its PDF is
\[
f_X(x)=\lambda e^{-\lambda x}u(x)
\]

\subsection*{Support}

\[
x \ge 0
\]

\subsection*{Probability Density Function (PDF)}

\[
f_X(x)=e^{-x}u(x)
\]

\subsection*{(a)}

\[
\Pr(3X<5)
\]

\[
= \Pr\left(X<\frac{5}{3}\right)
\]

\[
= \int_0^{5/3} e^{-x}\,dx
\]

\[
= 1-e^{-5/3}
\]

\subsection*{(b)}

\[
\Pr(3X<y)
\]

\[
= \Pr\left(X<\frac{y}{3}\right)
\]

\[
= \int_0^{y/3} e^{-x}\,dx
\]

\[
= 1-e^{-y/3}, \quad y\ge0
\]

\subsection*{(c) \quad Let $(Y=3X)$}

\[
F_Y(y)=\Pr(Y<y)=1-e^{-y/3}, \quad y\ge0
\]

\[
f_Y(y)=\frac{d}{dy}F_Y(y)
=\frac{1}{3}e^{-y/3}u(y)
\]

\section*{3. Laplace Random Variable}

\subsection*{Definition}

A random variable $(W)$ is Laplace distributed if
\[
f_W(w)=c e^{-2|w|}
\]

\subsection*{Support}

\[
-\infty < w < \infty
\]

\subsection*{(a)}

\[
\int_{-\infty}^{\infty} c e^{-2|w|}\,dw = 1
\]

\[
2c\int_0^\infty e^{-2w}\,dw = 1
\]

\[
2c\cdot\frac{1}{2} = 1
\Rightarrow c = 1
\]

\subsection*{(b)}

\[
\Pr(-1<W<2)
\]

\[
= \int_{-1}^0 e^{2w}\,dw + \int_0^2 e^{-2w}\,dw
\]

\[
= \left[\frac{1}{2}e^{2w}\right]_{-1}^{0}
+ \left[-\frac{1}{2}e^{-2w}\right]_{0}^{2}
\]

\[
= \frac{1}{2}(1-e^{-2}) + \frac{1}{2}(1-e^{-4})
\]

\subsection*{(c)}

\[
\Pr(W>0 \mid -1<W<2)
\]

\[
= \frac{\Pr(0<W<2)}{\Pr(-1<W<2)}
\]

\[
\Pr(0<W<2)
= \int_0^2 e^{-2w}\,dw
= \frac{1}{2}(1-e^{-4})
\]

\[
\Pr(W>0 \mid -1<W<2)
= \frac{\frac{1}{2}(1-e^{-4})}
{\frac{1}{2}(1-e^{-2})+\frac{1}{2}(1-e^{-4})}
\]

\section*{4. Gamma Random Variable}

\subsection*{Definition of the Gamma Function}

\[
\Gamma(x)=\int_0^\infty t^{x-1}e^{-t}\,dt
\]

\subsection*{Mean and Variance}

If $(X\sim\text{Gamma}(\alpha,\lambda))$,

\[
E[X]=\frac{\alpha}{\lambda}
\]

\[
\mathrm{Var}(X)=\frac{\alpha}{\lambda^2}
\]

\section*{5. Properties of the Gamma Function}

\subsection*{(a)}

\[
\Gamma(n)=(n-1)!
\]

\[
\Gamma(n)=\int_0^\infty t^{n-1}e^{-t}\,dt
\]

Using integration by parts repeatedly,
\[
\Gamma(n)=(n-1)\Gamma(n-1)
\]

With $(\Gamma(1)=1)$,
\[
\Gamma(n)=(n-1)!
\]

\subsection*{(b)}

\[
\Gamma(x+1)=x\Gamma(x)
\]

\[
\Gamma(x+1)=\int_0^\infty t^x e^{-t}\,dt
\]

Integration by parts gives:
\[
\Gamma(x+1)=x\Gamma(x)
\]

\subsection*{(c)}

\[
\Gamma\left(\frac{1}{2}\right)=\sqrt{\pi}
\]

\[
\Gamma\left(\frac{1}{2}\right)
=\int_0^\infty t^{-1/2}e^{-t}\,dt
\]

Using the Gaussian integral,
\[
\Gamma\left(\frac{1}{2}\right)=\sqrt{\pi}
\]

\section*{6. In-Class Activity: Gaussian Simulation and Gaussian Density Estimation}

(Google Colab)

\subsection*{Activity Overview}

\begin{itemize}
\item Gaussian simulation
\item Distribution visualization
\item Density estimation
\item Packet delay modeling
\end{itemize}

\subsection*{1. Motivation}

\begin{itemize}
\item Sensor measurements
\item Network packet delays
\item Image sensor noise
\end{itemize}

\subsection*{2. Gaussian Simulation (Known Parameters)}

\begin{lstlisting}[language=Python]
import numpy as np
import matplotlib.pyplot as plt

N = 10000
mu = 0
sigma = 1

samples = np.random.normal(mu, sigma, N)
\end{lstlisting}

\subsection*{3. Visualizing the Distribution}

\begin{lstlisting}[language=Python]
counts, bins = np.histogram(samples, bins=50, density=True)
centers = (bins[:-1] + bins[1:]) / 2

pdf = (1/(sigma*np.sqrt(2*np.pi))) * np.exp(-(centers-mu)**2/(2*sigma**2))

plt.bar(centers, counts)
plt.plot(centers, pdf)
plt.show()
\end{lstlisting}

\subsection*{4. Reality Check: Unknown Parameters}

\begin{itemize}
\item Only samples are available
\item Parameters unknown
\end{itemize}

\subsection*{5. Gaussian Density Estimation from Data}

\begin{lstlisting}[language=Python]
mu_hat = np.mean(samples)
sigma_hat = np.std(samples)
\end{lstlisting}

\[
\hat{f}(x)=
\frac{1}{\sigma_{\text{hat}}\sqrt{2\pi}}
e^{-\frac{(x-\mu_{\text{hat}})^2}{2\sigma_{\text{hat}}^2}}
\]

\subsection*{6. Application: Packet Delay in Networks}

\begin{itemize}
\item Gaussian delay simulation
\item Non-negative constraint
\item Histogram construction
\item Parameter estimation
\item Density plotting
\end{itemize}

\subsection*{7. Key Takeaways}

\begin{itemize}
\item Gaussian simulation uses randomized algorithms
\item Density estimation learns distributions from samples
\item Same methodology applies across applications
\end{itemize}

\end{document}
